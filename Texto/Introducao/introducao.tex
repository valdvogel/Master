%*****************************************************************************************
%*********************************** First Chapter ***************************************
%*****************************************************************************************

\chapter*[Introdução]{Introdução}
\addcontentsline{toc}{chapter}{Introdução}

Este documento e seu código-fonte são exemplos de referência de uso da classe
\textsf{abntex2} e do pacote \textsf{abntex2cite}. O documento 
exemplifica a elaboração de projetos de pesquisa produzidos
conforme a ABNT NBR 15287:2011 \emph{Informação e documentação - Projeto de
pesquisa - Apresentação}. 

A expressão ``Modelo canônico'' é utilizada para indicar que \abnTeX\ não é
modelo específico de nenhuma universidade ou instituição, mas que implementa tão
somente os requisitos das normas da ABNT. Uma lista completa das normas
observadas pelo \abnTeX\ é apresentada em \citeonline{abntex2classe}.
massas amadureceu \cite{Walton1990} 