
\documentclass[
	% -- opções da classe memoir --
	12pt,				% tamanho da fonte
	openright,			% capítulos começam em pág ímpar (insere página vazia caso preciso)
	twoside,			% para impressão em verso e anverso. Oposto a oneside
	a4paper,			% tamanho do papel. 
	% -- opções da classe abntex2 --
	%chapter=TITLE,		% títulos de capítulos convertidos em letras maiúsculas
	%section=TITLE,		% títulos de seções convertidos em letras maiúsculas
	%subsection=TITLE,		% títulos de subseções convertidos em letras maiúsculas
	%subsubsection=TITLE,	% títulos de subsubseções convertidos em letras maiúsculas
	% -- opções do pacote babel --
	%english,				% idioma adicional para hifenização
	%french,				% idioma adicional para hifenização
	%spanish,				% idioma adicional para hifenização
	%brazil,				% o último idioma é o principal do documento
	]{abntex2}


% ---
% Pacotes fundamentais 
% ---
\usepackage{polyglossia}
\setmainlanguage{brazil}
\usepackage{lmodern}			% Usa a fonte Latin Modern
%%\usepackage[T1]{fontenc}			% Selecao de codigos de fonte.
%%\usepackage[utf8]{inputenc}		% Codificacao do documento (conversão automática dos acentos)
\usepackage{indentfirst}			% Indenta o primeiro parágrafo de cada seção.
\usepackage{color}				% Controle das cores
\usepackage{graphicx}			% Inclusão de gráficos
\usepackage{microtype} 			% para melhorias de justificação
% ---

% ---
% Pacotes adicionais, usados apenas no âmbito do Modelo Canônico do abnteX2
% ---
\usepackage{lipsum}				% para geração de dummy text
% ---

% ---
% Pacotes de citações
% ---
%\usepackage[brazilian,hyperpageref]{backref}	 % Paginas com as citações na bibl
\usepackage[alf]{abntex2cite}				% Citações padrão ABNT

% --- 
% CONFIGURAÇÕES DE PACOTES
% --- 

% ---
% Configurações do pacote backref
% Usado sem a opção hyperpageref de backref
%\renewcommand{\backrefpagesname}{Citado na(s) página(s):~}
% Texto padrão antes do número das páginas
%\renewcommand{\backref}{}
% Define os textos da citação
%\renewcommand*{\backrefalt}[4]{
%	\ifcase #1 %
%		Nenhuma citação no texto.%
%	\or
%		Citado na página #2.%
%	\else
%		Citado #1 vezes nas páginas #2.%
%	\fi}%
% ---

% ---
% Informações de dados para CAPA e FOLHA DE ROSTO
% ---
\titulo{Uma proposta de método para Análise e Desenho de uma Arquitetura Orientada a Serviço}
\autor{José Valdvogel de Almeida Junior}
\local{São Paulo}
\data{2014}
\instituicao{%
  Pontíficia Universidade Católica - PUC-SP
  \par
  Tecnologias da Inteligência e Design Digital - TIDD
  \par
  Programa de Pós-Graduação}
\tipotrabalho{Tese (Mestrado)}





% O preambulo deve conter o tipo do trabalho, o objetivo, 
% o nome da instituição e a área de concentração 
\preambulo{
Dissertação apresentada à Pontíficia Universidade Católica de São Paulo para obtenção do título de Mestre em Tecnologias da Inteligência e Design Digital. 
Área de Concentração : Modelagem de Software 
Orientador: Prof. Dr. Ítalo Santiago Vega
}
% ---

% Configurações de aparência do PDF final

% alterando o aspecto da cor azul
%\definecolor{blue}{RGB}{41,5,195}

% informações do PDF
\makeatletter
\hypersetup{
     		%pagebackref=true,
		pdftitle={\@title}, 
		pdfauthor={\@author},
    		pdfsubject={\imprimirpreambulo},
	    	pdfcreator={LaTeX with abnTeX2},
		pdfkeywords={SOA}{Service}{Design}{TIDD}{projeto de pesquisa}, 
		colorlinks=false,       		% false: boxed links; true: colored links
    		linkcolor=blue,          		% color of internal links
    		citecolor=blue,        		% color of links to bibliography
    		filecolor=magenta,      	% color of file links
		urlcolor=blue,
		bookmarksdepth=4
}
\makeatother
% --- 

% --- 
% Espaçamentos entre linhas e parágrafos 
% --- 

% O tamanho do parágrafo é dado por:
\setlength{\parindent}{1.3cm}

% Controle do espaçamento entre um parágrafo e outro:
\setlength{\parskip}{0.2cm}  % tente também \onelineskip

% ---
% compila o indice
% ---
\makeindex
% ---

% ----
% Início do documento
% ----
\begin{document}


% Retira espaço extra obsoleto entre as frases.
\frenchspacing 

% ----------------------------------------------------------
% ELEMENTOS PRÉ-TEXTUAIS
% ----------------------------------------------------------
% \pretextual


% ---
% Capa
% ---
\imprimircapa
% ---

% ---
% Folha de rosto
% ---
\imprimirfolhaderosto
% ---


% ---
% NOTA DA ABNT NBR 15287:2011, p. 4:
%  ``Se exigido pela entidade, apresentar os dados curriculares do autor em
%     folha ou página distinta após a folha de rosto.''
% ---

% ---
% inserir lista de ilustrações
% ---
\pdfbookmark[0]{\listfigurename}{lof}
\listoffigures*
\cleardoublepage
% ---

% ---
% inserir lista de tabelas
% ---
\pdfbookmark[0]{\listtablename}{lot}
\listoftables*
\cleardoublepage
% ---

% ---
% inserir lista de abreviaturas e siglas
% ---
\begin{siglas}
  \item[SOA] Service Oriented Architecture
  \item[456] Isto é um número
  \item[123] Isto é outro número

\end{siglas}
% ---

% ---
% inserir lista de símbolos
% ---
\begin{simbolos}
  \item[$ \Gamma $] Letra grega Gama
  \item[$ \Lambda $] Lambda
  \item[$ \zeta $] Letra grega minúscula zeta
  \item[$ \in $] Pertence
\end{simbolos}
% ---

% ---
% inserir o sumario
% ---
\pdfbookmark[0]{\contentsname}{toc}
\tableofcontents*
\cleardoublepage
% ---



% ----------------------------------------------------------
% ELEMENTOS TEXTUAIS
% ----------------------------------------------------------
\textual

% ----------------------------------------------------------
% Introdução
% ----------------------------------------------------------
%\input{./Introducao/introducao.tex}



\chapter{SOA1}

Soa Capitulo 1 \cite{Wolf2003}

\chapter{SOA2}

%*****************************************************************************************
%*********************************** First Chapter ***************************************
%*****************************************************************************************

\begin{document}

%\chapter{Getting Started}  %Title of the First Chapter

Este documento e seu código-fonte são exemplos de referência de uso da classe
abntex2 e do pacote abntex2cite. O documento exemplifica a elaboração de projetos de
pesquisa produzidos conforme a ABNT NBR 15287:2011 Informação e documentação -
Projeto de pesquisa - Apresentação.
A expressão “Modelo canônico” é utilizada para indicar que abnTEX2 não é modelo
específico de nenhuma universidade ou instituição, mas que implementa tão somente os
requisitos das normas da ABNT. Uma lista completa das normas observadas pelo abnTEX2
é apresentada em ??).

\end{document}

Soa Capitulo 2


\bibliography{template.bib}

\end{document}
